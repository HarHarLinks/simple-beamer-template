\documentclass[12pt]{beamer}
%converts \items to more fancy bubbles
\usetheme{Boadilla}

%use some packages making it easier to write for germans and display various stuff
\usepackage[utf8]{inputenc}
\usepackage[german]{babel}
\usepackage{amsmath}
\usepackage{amsfonts}
\usepackage{amssymb}
\usepackage{graphicx}
\usepackage{multirow}

%disable navigation buttons in pdf
\usenavigationsymbolstemplate{}

%fill this in:
\author{} %insert author
\title{} %insert title
\subtitle{} %insert subtitle if applicable
\date{} %insert date (eg \today)
\institute{} %insert institute; multiple lines are possible

%create footer with author and current subtitle
\defbeamertemplate{footline}{author and page number}{%
	\usebeamercolor[fg]{page number in head/foot}%
	\usebeamerfont{page number in head/foot}%
	\hspace{1em}\insertshortauthor
	\hspace{10em}\insertsubtitle\hfill%
	\insertpagenumber\,/\,\insertpresentationendpage\kern1em\vskip2pt%
}
%use footer
\setbeamertemplate{footline}[author and page number]{}

%suppress the navigation bar
\beamertemplatenavigationsymbolsempty

%set captions to use a smaller font
\usepackage[font=small,labelformat=empty]{caption}

%display footnote only after reference gets active
\let\footnoterule\relax

%show a recap of the toc with current section highlighted at the start of every section
\AtBeginSection[]
{
  \begin{frame}
    \frametitle{Übersicht}
    \tableofcontents[currentsection]
  \end{frame}
}

\begin{document}

%this creates a titlepage for you using the information you entered previously
\begin{frame}[label=titlepage]
\titlepage
\end{frame}

%show the toc to introduce to the talk
\begin{frame}
	\frametitle{Übersicht}
    \tableofcontents
\end{frame}

%now follow some examples
%begin a section to have some basic structure
\section{First Section}
\begin{frame}{\insertsection}
%pause automatically generates a new frame so you can control when your content is displayed
	\pause %it basically works intuitively and also in nearly every possible case
%itemize is
	\begin{itemize}
	  \item a great way to present information without having too much text\pause
	  \item often used in talks
	\end{itemize}
\end{frame}

\begin{frame}{This frame has a manual title}{and also subtitle}
%columns and blocks are used to shape your frame into table-like sections
	\begin{columns}[t]
		\begin{column}{0.5\textwidth} %of course you can control the width

		\end{column}
		\begin{column}{0.5\textwidth}
		
		\end{column}
	\end{columns}
\end{frame}

%remember the toc returns at the start of each new section

\section{Second Section}
\subsection{First Subsection}
\begin{frame}{\insertsection}{\insertsubsection}
  You can also use subsections and naturally include them als frame subtitles
\end{frame}

\subsection{Second Subsection}
\begin{frame}{\insertsubsection}
  or just as titles of course
\end{frame}

\end{document}
